%encoding UTF-8
\author{Евгений Синельников, Игорь Чудов}
\city{Саратов}
\affiliation{Базальт СПО}
\projecttitle{GPOA}
\projecturl{\url{https://github.com/altlinux/gpupdate/}}
\title{Управление клиентами ОС ALT в рамках единой инфраструктуры
домена Active Directory}
\maketitle

\begin{abstract}
  В рамках доклада рассмотрена проблема проблема интеграции клиентов
  домена Active Directory под управлением ОС ALT. Рассмотрено ПО для
  работы с доменом и групповыми политиками в отечественной ОС ALT.
\end{abstract}


\section{Проблематика}

В силу неразвитости отечественного рынка ПО подавляющее число компаний
строят рабочую инфраструктуру используя продукт Microsoft Active
Directory. При выполнений требований импортозамещения отечественными
разработчиками операционных систем это является серьёзным блокирующим
фактором, так как требования зачастую пишутся в терминах и рамках
продукта AD (Active Directory).

Для преодоления этого ограничения и с целью удовлетворить потребности
администраторов, выполняющих миграцию инфраструктур, в компании
"Базальт СПО" были разработаны два продукта:

\begin{itemize}
\item \textbf{GPOA} - \textbf{Group Policy Object Applier} - для применения
настроек, полученных из групповых политик в среде UNIX-подобных ОС.
\item \textbf{ADMC} - \textbf{Active Directory Management Center} -
графический инструмент для управления доменом AD и групповыми политиками.
\end{itemize}

Кроме самих продуктов была проделана большая работа по устранению
несостыковок в различных компонентах дистрибутивов ОС ALT и расширению
их функционала, чтобы интеграция в домен с применением групповых
политик не сломала существующие инсталляции и могла быть произведена
плавно (или отменена в любой момент).

\section{Анализ проблемы}

Между методиками, принятыми для управления UNIX-like ОС, и методиками,
используемыми в среде ОС Windows лежит пропасть. Чтобы правильно оценить
потребности пользователей, пришлось провести обширный анализ различных
пользовательских отзывов и запросов, который впоследствии был
сформулирован в виде use-cases и целей.

Также пришлось произвести технический анализ устройства домена и
групповых политик, чтобы понять в каком виде и как они доставляются
для компьютеров и пользователей.

В связи с отсутствием глубокой экспертизы на первых этапах, продукт
пришлось строить "снизу вверх". Сначала разрабатывались небольшие
PoC для различных частей ОС, а потом это всё связывалось в единый
комплекс. Проект ещё должен пройти длительный путь развития, но уже
может применяться в простых конфигурациях.

\section{GPOA}

GPOA это утилита для вычитывания и применения настроек ОС для
пользователя или для машины в целом.

Это небольшая утилита, которая выполняет простую функцию вычитывания
групповых политик и их применения. Данное ПО состоит из трёх
компонентов:

\begin{itemize}
\item \textbf{Бэкэнда}, который занимается вычитыванием настроек из
различных источников.
\item \textbf{Хранилища}, которое занимается хранением настроек
независимо от их источника, и чем-то похоже на реестр Windows.
\item \textbf{Фронтенда}, который запускает модули, отвечающие за
применение различных настроек ко всей системе или к пользователю.
\end{itemize}

Не смотря на простое устройство кодовой базы, разработка этой части ПО
требует глубокой экспертизы в различных частях дистрибутивного решения,
в рамках которого ПО поставляется. Соответственно, распространение ПО
требует, чтобы дистрибутив представлял из себя комплексное
интегрированное решение с хорошей связью компонентов. В противном случае
использование может негативно отразиться на пользовательском восприятии.

\section{ADMC}

ADMC это графический инструмент, который выполняет функции, аналогичные
Microsoft RSAT (Remote System Administration Tool). В числе таковых
функций на текущий момент:

\begin{itemize}
\item Создание Organization Unit (OU)
\item Промотр списка введённых в домен машин
\item Редактирование списка пользователей
\end{itemize}

При разработке инструмента была сделана попытка без привязки к
требованиям обратной совместимости переосмыслить механизмы управления
доменом Active Directory и предоставить пользователям интерфейс, который
позволяет управлять доменом с наибольшей простотой и эффективностью.

Не смотря на частые запросы о реализации интерфейса управления в виде
WebUI было принято решение сделать классическое приложение с
использованием библиотеки Qt5 в целях обеспечения наибольшей
безопасности и функциональности.

\section{Библиография}

\begin{thebibliography}{9}
\bibitem{altwiki} ALT Linux Wiki \url{http://www.altlinux.org/Групповые_политики}
\end{thebibliography}


%%% Local Variables: 
%%% mode: latex
%%% TeX-master: "../main"
%%% End: 
