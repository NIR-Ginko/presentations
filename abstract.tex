%encoding UTF-8
\author{Игорь Чудов}
\city{Саратов}
\affiliation{BaseALT}
\projecttitle{SBCL}
\projecturl{\url{http://www.sbcl.org/}}
\title{Проблемы портирования SBCL на новые аппаратные платформы}
\maketitle

\begin{abstract}
  В докладе рассмотрена "проблема бутстрапа" реализации компилятора
  языка программирования Common LISP --- SBCL (Steel Bank Common LISP)
  в приложении к архитектуре e2k. Рассмотрены составные элементы данного
  ПО, этапы бутстрапа, а также некоторые подробности низкоуровневого
  устройства. Данная работа призвана объединить разрозенные знания об
  устройстве проекта.
\end{abstract}

\section{Проблематика}

Попытка безболезненной сборки компилятора SBCL на Elbrus 101-PC не
увенчалась успехом, в связи с чем было начато исследование проекта.

В процессе работы было обнаружено, что полноценное портирование на новые
архитектуры выполняется достаточно редко, а портирование на архитектуры
типа MIPS может выполняться по остаточному принципу о чём прямо написано
в коде. В результате в новых версиях программы поддержка конкретной
архитектуры может не гарантироваться.

Так как писать компилятор языка программирования на этом же языке
программирования является достаточно популярной тенденцией - при
возникновении новых архитектур возникает "проблема бутстрапа", которую
иногда называют проблемой "курицы и яйца". SBCL также подвержен данной
проблеме, но в относительно небольшой степени.

\section{Устройство и этапы сборки}

Бутстрап SBCL невозможен без существования интерпретатора Common LISP,
но первичная сборка может быть осуществлена без инструментов
кросс-компиляции. Для собственно инициализации сборки в таком случае
необходим компилятор C, ассемблер и линкер для целевой платформы.

Интерпретатор Common LISP может отсутствовать на целевой платформе, но
возможно применить его к дереву сборки перенеся на платформу, где
которой он уже имеется.

Простейший процесс сборки на новой платформе выглядит так:

Cross-compiling SBCL is easy; especially so if you arrange matters (e.g. using NFS)

so that host and target systems can see the same build directory. make.sh doesn't

cater for it directly, so you run each of the five scripts by hand. For example,

cross-compiling to PPC is something like


host$ export SBCL_ARCH=ppc

or


host$ export SBCL_ARCH=x86

ppc$ export SBCL_ARCH=x86


host$ export SBCL_XC_HOST=sbcl

ppc$ . ./find-gnumake.sh

ppc$ find_gnumake

ppc$ sh make-config.sh

host$ sh make-host-1.sh

ppc$ sh make-target-1.sh

host$ sh make-host-2.sh

ppc$ sh make-target-2.sh

ppc$ sh make-target-contrib.sh



target$ GNUMAKE=make sh ./make-config.sh

host$ GNUMAKE=gmake SBCL_ARCH='arch' sh ./make-config.sh

host$ scp target:sbcl-directory/local-target-features.lisp-expr .

host$ scp target:sbcl-directory/output/build-id.tmp output/

host$ SBCL_XC_HOST='lisp -batch' sh ./make-host-1.sh

host$ scp -r src/runtime/genesis src/runtime/ldso-stubs.S target:sbcl-directory/src/runtime/

target$ GNUMAKE=make sh ./make-target-1.sh

target$ scp src/runtime/sbcl.nm host:sbcl-directory/src/runtime/

target$ scp output/stuff-groveled-from-headers.lisp host:sbcl-directory/output/

host$ SBCL_XC_HOST='lisp -batch' sh ./make-host-2.sh

host$ scp output/cold-sbcl.core target:sbcl-directory/output/

target$ sh ./make-target-2.sh

target$ sh ./make-target-contrib.sh




\section{Благодарность}

\begin{itemize}
\item Bruce O'Neel
\item Charles Zhang
\end{itemize}

В тексте могут встречаться: 
\begin{itemize}
\item имена файлов, команд и т. п. в неизменном виде \Q{verbatim text
    with ^_@&\%\#\${}\\};
\item URL \url{http://somewhere.org};
\item англоязычные вставки \EN{english phrase};
\item текстовые выделения \emph{курсив}.
\end{itemize}

\section{Могут встречаться подразделы}

Требования к иллюстрациям: 
\begin{itemize}
\item \emph{Чёрно-белые}. Если оригиналы у Вас цветные, то стоит
  самостоятельно сконвертировать в оттенки серого и посмотреть, что
  стало с контрастностью и насыщенностью, в случае необходимости
  подправить.
\item \emph{Ширина} изображений не должна превышать 110\,мм,
  \emph{высота} "--- не более 160\,мм.
\item \emph{Растровые} -- в формате без потери качества (TIFF), 300 dpi.
\item \emph{Векторные} -- в векторном формате, трансформируемом без потерь в
  eps (лучше всего сразу в eps).
\item Если на изображении есть \emph{текст}, стоит проверить, что при
  данном размере (на печати) буквы не слишком маленькие.
\end{itemize}

% Вставка иллюстраций.
% \begin{figure}
%   \centering
%   \includegraphics{example.eps}
% \caption{Подпись к иллюстрации (необязательно). Точка в конце не ставится}
% \end{figure}

Оформление ссылок\cite{brooks}. Простое оформление библиографии
приведено ниже, однако не возбраняется и даже приветствуется
использование BibTeX.

\begin{thebibliography}{9}
\bibitem{brooks} \textit{Брукс Ф.} Мифический человеко-месяц,
  или как создаются программные системы. \url{http://www.lib.ru/CTOTOR/BRUKS/mithsoftware.txt}
\end{thebibliography}


%%% Local Variables: 
%%% mode: latex
%%% TeX-master: "../main"
%%% End: 
